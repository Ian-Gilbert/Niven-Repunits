\documentclass{amsart}
\usepackage{myStyle}

\begin{document}

\title{Classifying Niven Repunits}
\author{Ian Gilbert}
\date{\today}

\begin{abstract}
A positive integer which is divisible by its digital sum is called a Niven number. A Niven repunit is a Niven number such that its decimal representation is all ones, for example the repunits 1, 111, 111111111, and 111111111111111111111111111 (27 ones) are the first four Niven repunits. Here, we will give a complete characterization of such integers. In addition, it will be pointed out how all Niven repunits can be constructed from a certain list of primes.

\textit{Note:} This paper is essentially a recreation of \cite{nivenrepunits}, because either I am misinterpreting the paper, or they are really bad mathematicians.
\end{abstract}

\maketitle

\section*{Terminology}

\begin{description}
\item[Niven number] A Niven number is a positive integer that is divisible by the sum of its digits.
\item[Repnumber] A repnumber is an integer where all the digits are the same, i.e. 444.
\item[Repunit] Repunits are a subset of repnumbers: an integer where all the digits are specifically 1.
\item[Niven Repunit] A Niven repunit combines the definitions of a Niven number and a repunit: an integer where all the digits are 1, and it is divisible by the sum of its digits.
\end{description}

\section{Introduction}

Many problems in number theory originate in very simple problem statements. For example, the inspiration for this project comes from something called `the 37 trick'. First, choose any digit 1-9, call it d. Now, compute $ddd/(d+d+d)$ where $ddd$ describes the 3-digit repnumber with digits $d$, such as 111, 222, or 333. Regardless of the digit you chose, you should get 37 as your answer. But why does this work? Observe the following: \[\frac{ddd}{d+d+d}=\frac{d(111)}{d(1+1+1)}=\frac{111}{1+1+1}=37\] In short, the digit you chose cancels out, and you are left with $111/(1+1+1)$, which just happens to equal 37.

But why does the number have to be three digits long? Does the trick still work with different length numbers? Short answer, yes. If you choose to use a four digit number, the chosen digit will still cancel, and you will always end up with the number $1111/(1+1+1+1)=277.75$. So the 37 trick will always work, but you will not always get an integer. This paper tries to answer the question of when the extrapolated 37 trick results in an integer.

\section{Bookkeeping}

First of all, we need a formal definition for a Niven repunit. Notice that raising $10^n$ gives us the smallest possible $n+1$ digit integer. If we subtract 1 from that, we get an $n$-digit repunit with digits 9. Dividing this by 9 then gives us the $n$-digit repunit. We can formalize this definition as the following.

\begin{definition}
\label{repunit}
The $n$-digit repunit is given by
\begin{equation}
\label{repuniteq}
    R(n)=\frac{10^n-1}{9}
\end{equation}
\end{definition}

From this, we get our definition for a Niven repunit.

\begin{definition}
\label{nivenrepunit}
A Niven repunit is a repunit divisible by the sum of its digits. Therefore a repunit $R(n)$ is Niven if and only if $R(n)\equiv0\pmod{n}$.
\end{definition}

Before we continue, we need to mention a useful Lemma.

\begin{lemma}
\label{useful lemma}
Let $a$, $b$, $m$, and $n$ by positive integers, and let $k$ be a non-negative integer. If $a\equiv b\pmod{m^n}$, then $a^{m^k}\equiv b^{m^k}\pmod{m^{k+n}}$.

\begin{proof}
Let $a,b,m,n\in\Z^+$, let $k$ be a non-negative integer, and suppose that $a\equiv b\pmod{m^n}$. The proof follows by induction on $k$.
\begin{description}
\item[Base Case] Let $k=0$, and note that $m^{k+n}=m^n$. Then $a^{m^k}\equiv a^1\equiv a\equiv b\equiv b^1\equiv b^{m^k}\pmod{m^{k+n}}$.

\vspace{5pt}

\item[Induction Hypothesis] Suppose that $a^{m^k}\equiv b^{m^k}\pmod{m^{k+n}}$, and consider $a^{m^{k+1}}-b^{m^{k+1}}$. Note that this can be expanded to \[a^{m^{k+1}}-b^{m^{k+1}}=\left(a^{m^k}-b^{m^k}\right)\left[\left(a^{m^k}\right)^{m-1}+\left(a^{m^k}\right)^{m-2}\left(b^{m^k}\right)+\cdots+\left(b^{m^k}\right)^{m-1}\right]\] By our Induction Hypothesis, it follows that 
\end{description}
\end{proof}
\end{lemma}

test of the 

\nocite{*}
\printbibliography[title=References]
\end{document}